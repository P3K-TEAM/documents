\documentclass{article}

\usepackage[margin=2cm]{geometry}
\usepackage[utf8]{inputenc}
\usepackage{hyperref}
\usepackage{biblatex}

% Title
\newcommand{\documenttitle}
{Verziovanie zdrojového kódu}

% Subtitle
\newcommand{\documentsubtitle}
{Metodika}


\begin{document}

    \newcommand{\fiitmark}{%
\large
\textnormal{%
Slovenská technická univerzita v Bratislave \\
\vspace*{.25cm}
Fakulta informatiky a informačných technológií} \\
}

\begin{titlepage}
        \begin{center}
            \vspace*{.5cm}
            \fiitmark

            \vfill
            \huge
            \textbf{\documenttitle} \\

            \vspace*{.25cm}
            \large
            \textbf{\documentsubtitle} \\

            \vspace*{.25cm}
            \textnormal{P3K Team (Tím 10)} \\

        \end{center}

        \vfill

        \begin{flushleft}
            \textbf{Vedúci Projektu}: Ing. Juraj Petrík \\
            \textbf{Akad. rok:} 2020/2021
        \end{flushleft}
\end{titlepage}


    \begin{table}[h]
        \begin{tabular}{ll}
            \textbf{Autor:} & Bc. František Gič \\
            \textbf{Dátum vytvorenia:} & 5.11.2020 \\
            \\
            \textbf{Autor poslednej zmeny:} &  \\
            \textbf{Dátum poslednje zmeny:} &  \\
            \hline
        \end{tabular}
        \label{tab:grades}
    \end{table}

    \section*{Úvod}
        \textnormal{%
        Verziovanie zdrojového kódu je jedna z najdôležitejších vecí pre prácu vývojárov v tíme. \\
        Pre verziovanie používame verziovací nástroj \emph{Git} a kód nahrávame na \emph{GitHub}.
        }

    \section*{Repozitáre}

        \textnormal{%
        Všetky repozitáre sú umiestnené v jednej organizácií. Vytvorenie nového repozitára je na spoločnej dohode medzi vývojármi.
        Organizácia je dostupná na nasledujúcom linku: \href{https://github.com/P3K-TEAM/}{https://github.com/P3K-TEAM/}
        }

    \section*{Vytváranie vetiev}

        \textnormal{Každá vetva (branch) musí byť správne pomenovaná v tvare \emph{\{Skratka JIRA projektu\}-\{Číslo úlohy\}-\{názov tasku v lowercase, rozdelený pomlčkami\}}} \\
        \smallskip
        \textbf{Napríklad:} \emph{FRON-17-define-git-methodology}

    \section*{Vytváranie commitov a písanie commit messages}

        \vspace*{.25cm}
        \subsection*{Kedy je správne vytvoriť commit?}

            \textnormal{%
            Commit označuje stav codebase v repozitári, ktorý je stabilný a každý z commitov prináša ucelenú zmenu do kódu. \\
            Vývojár by mal commitovať ucelenú zmenu ako celok. V prípade že by prišlo do stavu, že by v jednej commit message mal rozdeliť funkcie čiarkou (napr. \emph{Add foo, create bar, implement foobar}), je správne rozdeliť tento commit na dané celky a commit-núť samostatne. \\
            }

            \noindent \textnormal{%
            Veľký počet commitov nemusí byť zlá vec, pokiaľ dávajú zmysel. Niekedy aj zmena jedného písmena pri HOTFIX-e je dôvodom na commit. V prípade že daná úloha je komplexného charakteru, rozdelenie úlohy na menšie časti v podobe commitov pomôže iným vývojárom ľahšie revidovať kód po jednotlivých commit-och.
            }

        \vspace*{.5cm}
        \subsection*{Ako napísať dobrý commit message?}

            \textnormal{Commit message, teda správa commitu, je správa, ktorá vysvetľuje ostatným vývojárom, čo daný commit robí v prípade, že naň nastavia hlavičku verziovacieho systému.}

            \begin{itemize}
                \item Vo všeobecnosti by mala byť správa krátka (do 50 znakov) a výstižná.
                \item Znenie správy musí byť v anglickom jazyku. Ustáleným konsenzom medzi vývojármi je písanie commit message v infinitíve slovesa.
                \item Memotechnickou pomôckou je pridanie na začiatok slovného spojenia \emph{This commit will ...}.\\\\
                \textbf{Príklady:}
                \begin{itemize}
                    \item This commit will ... Setup TailwindCSS framework into the project
                    \item This commit will ... Implement unit tests
                    \item This commit will ... Refactor XYZ component
                \end{itemize}
            \end{itemize}



            \noindent \textnormal{%
            Viac o commit messagoch je dobré si prečítať na
            \href{https://www.freecodecamp.org/news/writing-good-commit-messages-a-practical-guide/}{FreeCodeCamp - Writing good commit messages - A practical guide} \\\\
            \textbf{Poznámka:} Takisto ako pred vetvu, analogicky aj pred commit message sa vkladá identifikátor JIRA úlohy. Rozdiel voči názvu vety je, že commit message sa píše s veľkým písmenom a slová sú rozdelené medzerami. \\
            \textbf{Príklad:} \emph{PW-13 Refactor UploadFile component}
            }
    

    \pagebreak
    \section*{Kontribúcia a spájanie vetiev}

        Každá zmena v kóde počas úlohy musí byť namapovaná na JIRA úlohu.
        V prípade, že je daná úloha hotová, je nahratá na GitHub server (\emph{git push}).
        Na GitHub serveri daný vývojár vytvorí \textbf{Pull Request} s danými zmenami.
        Následne kontaktuje iných vývojárov v tíme, a požiada ich (použitím oficiálneho komunikačného kanálu - Slacku),
        aby urobili revíziu kódu (Code review). \\

        \noindent \textbf{Poznámka: Kontaktovanie iných developerov je výlučne zodpovednosť autora Pull Request-u! Nezabúdajme, že úloha je hotová, až v prípade, že sú zmeny zlúčené s \emph{master} vetvou.} \\

    \begin{itemize}
        \item Názov pull requestu je zhodný s JIRA úlohou (\emph{PW-13 Refactor UploadFile component})
        \item V prípade, že chce vývojár rozdeliť revíziu kódu na časti alebo chce vytvoriť pull request predčasne (keď pull request ešte nie je vhodný na revíziu), označí merge request prefixom \emph{WIP:}
        \item Vytvorením pull requestu sa stav úlohy v JIRA mení na \emph{Code review}
        \item V prípade že je to nutné, pull request obsahuje aj dokumentáciu k danej zmene - buď na Confluence, alebo priamo v repozitári v priečinku \emph{docs}
    \end{itemize}

    \bigskip

    \textbf{Merge request môže byť zlúčený do master vetvy:}
    \begin{itemize}
        \item ak bol schválený aspoň iným vývojárom
        \item CI/CD Pipeline je v stave zelená (passed)
        \item všetky názvy commitov obsahujú konvenciu
        \item vytvorením pull requestu sa stav úlohy v JIRA mení na \emph{Code review}
        \item ak je to nutné, pull request obsahuje aj dokumentáciu k danej zmene - buď na Confluence, alebo priamo v repozitári v priečinku \emph{docs}
    \end{itemize}
\end{document}
