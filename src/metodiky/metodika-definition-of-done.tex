\documentclass{article}

\usepackage[margin=2cm]{geometry}
\usepackage[utf8]{inputenc}
\usepackage{hyperref}

% Title
\newcommand{\documenttitle}
{Definition of Done}

% Subtitle
\newcommand{\documentsubtitle}
{Metodika}

\begin{document}

    \newcommand{\fiitmark}{%
\large
\textnormal{%
Slovenská technická univerzita v Bratislave \\
\vspace*{.25cm}
Fakulta informatiky a informačných technológií} \\
}

\begin{titlepage}
        \begin{center}
            \vspace*{.5cm}
            \fiitmark

            \vfill
            \huge
            \textbf{\documenttitle} \\

            \vspace*{.25cm}
            \large
            \textbf{\documentsubtitle} \\

            \vspace*{.25cm}
            \textnormal{P3K Team (Tím 10)} \\

        \end{center}

        \vfill

        \begin{flushleft}
            \textbf{Vedúci Projektu}: Ing. Juraj Petrík \\
            \textbf{Akad. rok:} 2020/2021
        \end{flushleft}
\end{titlepage}


    \noindent \textnormal{Z hľadiska efektívneho vývoja je potrebné definovať kritéria, ktoré musia byť splnené, aby sme danú user story mohli považovať za splnenú.}

    \section*{User story}

    \textnormal{User story môže byť označený ako “done” iba v prípade, že spĺňa nasledujúce kritéria:}

    \begin{itemize}
        \item Všetky tasky sú vykonané.
        \item User story je otestovaný a všetky testy zbehli úspešne.
        \item Dosiahnutá funkcionalita spĺňa definované akceptačné kritéria.
        \item Prebehlo code review.
        \item Všetky zmeny a konfigurácie sú dokumentované.
    \end{itemize}

    \section*{Šprint}

    \textnormal{Šprint môže byť označený ako “done” iba v prípade, že spĺňa nasledujúce kritéria:}

    \begin{itemize}
        \item Všetky úlohy v danom šprinte sú hotové.
        \item Product backlog bol aktualizovaný.
        \item Prebehli sprint review a retrospektíva šprintu.
        \item Šprint je zdokumentovaný.
        \item Šprint bol akceptovaný vlastníkom produktu.
    \end{itemize}


    \section*{Nasadenie do produkcie}

    \textnormal{Nasadenie do produkcie môže byť vykonané iba v prípade, že spĺňa nasledujúce kritéria:}

    \begin{itemize}
        \item Všetky user story sú hotové na základe definition of done.
        \item Prostredie je pripravené na nasadenie.
        \item Produkt je označený ako pripravený na nasadenie od členov tím a zákazníka.
    \end{itemize}

\end{document}
