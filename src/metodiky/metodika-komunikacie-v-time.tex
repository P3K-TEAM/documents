\documentclass{article}

\usepackage[margin=2cm]{geometry}
\usepackage[utf8]{inputenc}
\usepackage{hyperref}

% Title
\newcommand{\documenttitle}
{Komunikácia v tíme}

% Subtitle
\newcommand{\documentsubtitle}
{Metodika}

\begin{document}

    \newcommand{\fiitmark}{%
\large
\textnormal{%
Slovenská technická univerzita v Bratislave \\
\vspace*{.25cm}
Fakulta informatiky a informačných technológií} \\
}

\begin{titlepage}
        \begin{center}
            \vspace*{.5cm}
            \fiitmark

            \vfill
            \huge
            \textbf{\documenttitle} \\

            \vspace*{.25cm}
            \large
            \textbf{\documentsubtitle} \\

            \vspace*{.25cm}
            \textnormal{P3K Team (Tím 10)} \\

        \end{center}

        \vfill

        \begin{flushleft}
            \textbf{Vedúci Projektu}: Ing. Juraj Petrík \\
            \textbf{Akad. rok:} 2020/2021
        \end{flushleft}
\end{titlepage}


    \begin{table}[h]
        \begin{tabular}{ll}
            \textbf{Autor:} & Bc. Karin Maliniaková \\
            \textbf{Dátum vytvorenia:} & 21.10.2020 \\
            \\
            \textbf{Autor poslednej zmeny:} &  \\
            \textbf{Dátum poslednje zmeny:} &  \\
            \hline
        \end{tabular}
        \label{tab:grades}
    \end{table}

    \section*{Komunikačné nástroje}

        \textnormal{Vzhľadom na situáciu, v ktorej sme sa ocitli v súvislosti s COVID-19 je na našej fakulte dištančná výučba čo spôsobilo, že komunikácia tímu prebieha online. Keďže je viacero rôznych dôvodov na komunikáciu v tíme, preto sme si zvolili aj viacero komunikačných kanálov, kde spolu komunikujeme.}

        \subsection*{Google Meets}

            \textnormal{Tento nástroj využívame ako náhradu osobných stretnutí, ktoré by za normálnych okolností normálne prebiehali v tíme. Prebiehajú tu teda pravidelné stretnutia v tíme bez vedúceho minimálne raz týždenne a tiež aj stretnutia s vedúcim projektu raz do týždňa.}


        \subsection*{Slack}

            \textnormal{Na komunikáciu medzi členmi tímu a aj komunikáciu s vedúcim projektu využívame Slack. Je to primárny nástroj našej komunikácie o projekte okrem hovorov. Komunikácia na Slacku je organizovaná do niekoľkých kanálov, aby sme sa vedeli rýchlejšie vrátiť k nejakej konkrétnej informácii a pre lepšiu prehľadnosť správ.}

            \begin{itemize}
                \item \textbf{\#general} \\
                \textnormal{Kanál určený na všeobecnú komunikáciu tímu o projekte, ktorú nie je vhodné začleniť do žiadneho iného kanálu.}
                \item \textbf{\#random} \\
                \textnormal{Kanál určený na neformálnu komunikáciu medzi členmi tímu a vedúcim, ktorá sa netýka vypracovania projektu.}
                \item \textbf{\#literatura} \\
                \textnormal{Kanál určený na zdieľanie literatúry, ktoré je potrebné/dobré si preštudovať, kvôli nejakej úlohe.}
                \item \textbf{\#dolezite-linky} \\
                \textnormal{Kanál určený na zdieľanie odkazov na rýchly prístup k poznámkam, online nástrojom, ktoré používame v tíme.}
                \item \textbf{\#dev-team-web} \\
                \textnormal{Kanál určený na komunikáciu o prezentačnej webovej stránke tímu.}
            \end{itemize}

        \subsection*{Discord}

            \textnormal{Využívaný na spoločné pracovné stretnutia, kde spoločne niečo programujeme alebo sa navzájom vzdelávame. Umožňuje nám spolu volať a bez problémov zdieľať obrazovku.}

        \subsection*{Email}

            \textnormal{Na oficiálnu komunikáciu tímu s ďalšími stranami využívame skupinový email \href{mailto:tim10@fiitgooglegroups.com}{tim10@fiitgooglegroups.com}. Tiež tento email využívame pri plánovaní Meetingových hovorov, aby pozvánku dostali všetci tímu a na nikoho sa nezabudlo.}

\end{document}
