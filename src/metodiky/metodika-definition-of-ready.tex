\documentclass{article}

\usepackage[margin=2cm]{geometry}
\usepackage[utf8]{inputenc}
\usepackage{hyperref}

% Title
\newcommand{\documenttitle}
{Definition of Ready}

% Subtitle
\newcommand{\documentsubtitle}
{Metodika}

\begin{document}

    \newcommand{\fiitmark}{%
\large
\textnormal{%
Slovenská technická univerzita v Bratislave \\
\vspace*{.25cm}
Fakulta informatiky a informačných technológií} \\
}

\begin{titlepage}
        \begin{center}
            \vspace*{.5cm}
            \fiitmark

            \vfill
            \huge
            \textbf{\documenttitle} \\

            \vspace*{.25cm}
            \large
            \textbf{\documentsubtitle} \\

            \vspace*{.25cm}
            \textnormal{P3K Team (Tím 10)} \\

        \end{center}

        \vfill

        \begin{flushleft}
            \textbf{Vedúci Projektu}: Ing. Juraj Petrík \\
            \textbf{Akad. rok:} 2020/2021
        \end{flushleft}
\end{titlepage}


    \begin{table}[h]
        \begin{tabular}{ll}
            \textbf{Autor:} & Bc. Karin Maliniaková \\
            \textbf{Dátum vytvorenia:} & 21.10.2020 \\
            \\
            \textbf{Autor poslednej zmeny:} &  \\
            \textbf{Dátum poslednje zmeny:} &  \\
            \hline
        \end{tabular}
        \label{tab:grades}
    \end{table}

    \section*{User story}

    \textnormal{User story môže byť označený ako “ready” iba v prípade, že spĺňa všetky nasledujúce kritéria:}

    \begin{itemize}
        \item User story nie je závislý žiadnou inou story, ktorá sa nebude nachádzať v pripravenom šprinte.
        \item User story sa môže meniť len do začiatku šprintu. Ako náhle začne šprint a user story sa nachádza v šprinte, nie je možné ju meniť, aby nedošlo k problémom v sprint review.
        \item User story je jasne definovaný a akceptovaný každým členom tímu. Z opisu musí byť jasné, čo má byť cieľom user story aby každý člen, tímu mal rovnakú predstavu.
        \item User story musí mať definované akceptačné kritéria a musia byť schválené tímom.
        \item User story musí byť vždy prediskutovaný so zákazníkom/product ownerom.
    \end{itemize}

\end{document}
