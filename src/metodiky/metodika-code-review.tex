\documentclass{article}

\usepackage[margin=2cm]{geometry}
\usepackage[utf8]{inputenc}
\usepackage{hyperref}

% Title
\newcommand{\documenttitle}
{Code review}

% Subtitle
\newcommand{\documentsubtitle}
{Metodika}

\begin{document}

    \newcommand{\fiitmark}{%
\large
\textnormal{%
Slovenská technická univerzita v Bratislave \\
\vspace*{.25cm}
Fakulta informatiky a informačných technológií} \\
}

\begin{titlepage}
        \begin{center}
            \vspace*{.5cm}
            \fiitmark

            \vfill
            \huge
            \textbf{\documenttitle} \\

            \vspace*{.25cm}
            \large
            \textbf{\documentsubtitle} \\

            \vspace*{.25cm}
            \textnormal{P3K Team (Tím 10)} \\

        \end{center}

        \vfill

        \begin{flushleft}
            \textbf{Vedúci Projektu}: Ing. Juraj Petrík \\
            \textbf{Akad. rok:} 2020/2021
        \end{flushleft}
\end{titlepage}


    \section*{Pull Request}

    Pull Request (GitHub), alebo Merge Request (GitLab), ďalej už len PR / MR, je žiadosť o vzájomnú kontrolu
    kódu osobami inými ako je autor danej časti zdrojového kódu.

    \subsection*{Kedy vytvoriť PR?}

    Pull request je možné vytvoriť v akomkoľvek stave vypracovania úlohy. V prípade že úloha nie je dokončená, sa na začiatok
    názvu PR prilepí skratka \textbf{WIP:} a pokračuje názvom PR.
    Skratka WIP (Work in progress) ostatným developerom značí, že daná časť kódu ešte nie je vhodná na code review.

    \subsection*{Názov pull requestu}

    Názov pull requestu by mal začínať skratkou projektu v JIRE a referenčným číslom. Po tomto nasleduje názov JIRA úlohy.\\\\
    \textbf{Príklad:} \emph{PW-13 Refactor UploadFile component}

    \subsection*{Obsah PR}

    PR Obsahuje iba tie zmeny, ktoré sú relevantné s JIRA úlohou, ktorej sa daný PR týka.\\

    \noindent Pull request obsahuje primárne zdrojové kódy. V prípade, že autor potrebuje na vzdialený server nahrať aj väčšie
    súbory (rozumej 1MB a viac), použije technológiu \href{https://git-lfs.github.com/}{Git LFS (Large File Storage)}.\\

    \noindent Ak je to nutné, pull request obsahuje aj dokumentáciu k danej zmene vo formáte \href{https://www.markdownguide.org/}{MarkDown}.\\

    \noindent V prípade, ak je špecifikované v rámci JIRA úlohy, že úloha má obsahovať aj testy k danej funkcionalite,
    pull request obsahuje kód spoločne s testami (Unit, E2E). V prípade že kód je subjektom testov, kód bol nimi zároveň otestovaný.


    \section*{Prehliadka kódu (code review)}

    Vytvorením pull requestu sa stav úlohy v JIRA mení na \emph{Code review}.
    V tomto momente autor zmien kontaktovuje iných developerov (použitím oficiálneho
    komunikačného kanálu - Slacku) a požiadať ich o prehliadku kódu.\\

    \noindent \textbf{Poznámka: Kontaktovanie iných developerov je výlučne zodpovednosť autora Pull Request-u!
    Nezabúdajme, že úloha je hotová, až v prípade, že sú zmeny zlúčené s \emph{master} vetvou.} \\

    \subsection*{Kontrola kódu zahrňuje}

    \noindent Pokiaľ bol iný developer požiadaný o code review, skontroluje kód:

    \begin{itemize}
        \item V prípade že nejde o jednoduchú zmenu, reviewer je povinný si danú vetvu lokálne odzrkadliť a zmeny pull
        requestu manuálne skontrolovať.
        \item V prípade že ide o zmenu vo vizuále webstránky ktorá bola predtým nadizajnovaná UX dizajnérom, reviewer je povinný
        porovnať, či sa daná implementácia zhoduje s navrhnutým dizajnom.
        \item Skontrolovanie, či úloha spĺňa všetky akceptačné kritériá JIRA úlohy.
    \end{itemize}

    \noindent Nezabúdajme, že za kvalitu kódu zodpovedá výhradne vývojár ktorý robil code review. Majme na mysli, že nekvalitný kód
    prináša zvýšené kapacity na úpravu v budúcnosti, a pre to dbajme na kvalitný code review.

    \pagebreak

    \subsection*{Kvalita kódu}

    \begin{itemize}
        \item Kód je písaný čitateľne (názvy premenných, optimálna dĺžka riadku, logické a syntaktické chyby)
        \item V prípade že ide o funkcionalitu, ktorá nie je jednoznačná, kód obsahuje komentáre pre takú časť kódu.
        \item Kód je validný a správne naformátovaný.
        \item V prípade že sú súčasťou kódu aj statické texty pre užívateľa, dodržujú správnu gramatiku daného jazyka.
        \item Opakujúci sa kód je vyňatý do samostatnej metódy / funkcie.
        \item Vývojár použil funkcionalitu ktorá existuje v niektorej z knižníc (Do not reinvent the wheel).
        \item Z názvu funkcií a metód sa dá jednoznačne pochopiť čo daná funkcia robí a čo vracia.
    \end{itemize}

    \subsection*{Ukončenie code review}

    \noindent V prípade, že vývojár našiel chybu v kóde, alebo potrebuje niečo prediskutovať s vývojárom, primárne používa komentáre
    v UI GitHub-u. V taktomto prípade po ukončení review zvolí možnosť \textbf{Request changes}. \\

    \noindent V prípade, že kód splňa požiadavky uvedené v tejto metodike, reviewer zvolí možnosť \textbf{Approve}. \\

    \noindent Udelením súhlasu iného vývojára je autorovi pull requestu dovolené pomocou GitHubu zlúčiť daný pull request s master vetvou
    použitím tlačidla \emph{Merge}, a následne nastaví status JIRA úlohy do statusu \emph{Done}.

\end{document}
