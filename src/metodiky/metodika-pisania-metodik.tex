\documentclass{article}

\usepackage[margin=2cm]{geometry}
\usepackage[utf8]{inputenc}
\usepackage[T1]{fontenc}

% Title
\newcommand{\documenttitle}
{Písanie a zmena metodík}

% Subtitle
\newcommand{\documentsubtitle}
{Metodika}

\begin{document}

    \newcommand{\fiitmark}{%
\large
\textnormal{%
Slovenská technická univerzita v Bratislave \\
\vspace*{.25cm}
Fakulta informatiky a informačných technológií} \\
}

\begin{titlepage}
        \begin{center}
            \vspace*{.5cm}
            \fiitmark

            \vfill
            \huge
            \textbf{\documenttitle} \\

            \vspace*{.25cm}
            \large
            \textbf{\documentsubtitle} \\

            \vspace*{.25cm}
            \textnormal{P3K Team (Tím 10)} \\

        \end{center}

        \vfill

        \begin{flushleft}
            \textbf{Vedúci Projektu}: Ing. Juraj Petrík \\
            \textbf{Akad. rok:} 2020/2021
        \end{flushleft}
\end{titlepage}


    \begin{table}[h]
        \begin{tabular}{ll}
            \textbf{Autor:} & Bc. Karin Maliniaková \\
            \textbf{Dátum vytvorenia:} & 21.10.2020 \\
            \\
            \textbf{Autor poslednej zmeny:} & Bc. Karin Maliniaková \\
            \textbf{Dátum poslednje zmeny:} & 19.11.2020 \\
            \hline
        \end{tabular}
        \label{tab:grades}
    \end{table}

    \section*{Písanie metodiky}

        \textnormal{%
        Tému metodiky volíme vždy, tak aby daná metodika bola užitočná a aby zefektívnila nejaký postup alebo zadefinovala určité pojmy. Pri písaní metodiky nie je dôležitý rozsah. Ak je niečo komplikovanejšie a vyžaduje si to podrobnejšie spracovanie, metodika môže mať aj niekoľko strán. Avšak aj sa jedná len o niečo malé stačí aj pár viet. \\\\
        Metodiku môže vytvoriť ktokoľvek z tímu v Latexu, podľa základnej šablóny. Dokument s metodikou bude napísaný v slovenskom jazyku, prekonvertovaný do pdf a po konzultácii s tímom aj na tímovom prezentačnom webu. 
        }

        \subsection*{Šablóna metodiky}

            \textnormal{Zachovávame úvodný formát definovaný v šablóne. Osoba, ktorá vytvára metodiku sa zapíše ako “autor” a zapíše “dátumu vytvorenia”, aby sme mali prehľad kedy metodika vznikla.}

        \subsection*{Formát textu}

            \textnormal{Pri písaní dokumentu by mali byť zachované nasledujúce pravidlá formátovania textu:}

            \begin{itemize}
                \item Nadpis - section*
                \item Normálny text - textnormal
                \item Odrážky - item
                \item formátovanie nadpisov musí po sebe logicky nadväzovať (Vzostupne)
                \item •	vo vhodných situáciách na sprehľadnenie textu využívame \emph{Zoznam s odrážkami} (itemize) alebo \emph{Číslovaný zoznam} (enumerate)
                \item linky na web musia byť uvedené pomocou hypertextových odkazov
            \end{itemize}

            \noindent \textnormal{Pri štrukturovaní dokumentu využívame rôzne úrovne nadpisov, podľa logickej nadväznosti daných častí textu}


        \subsection*{Obrázky}

            \textnormal{Pri vkladaní obrázkov používame nasledujúce pravidlá vkladania obrázkov:}

            \begin{itemize}
                \item obrázky sú vložené v takej veľkosti, aby neobmedzovali prehľadnosť textu
                \item text v obrázkoch musí byť čitateľný
            \end{itemize}

        \subsection*{Štruktúra metodiky}

            \textnormal{Štruktúra metodiky musí spĺňať nasledujúce pravidlá:}

            \begin{itemize}
                \item metodika musí obsahovať hierarchickú štruktúru, ktorá bude znázornená úrovňami nadpisov
                \item každý nadpis (resp. kapitola), ak je to možné, by mal obsahovať popis, čo sa pod ním bude nachádzať
                \item nadpisy musia byť stručné a výstižné
                \item v prípade, že je metodika dlhšia ako 2 strany a obsahuje viacero nadpisov, musí mať na začiatku súboru uvedený obsah pre lepšiu prehľadnosť dokumentu
            \end{itemize}

    \section*{Zmena metodík}

        \textnormal{Vykonať zmenu v metodike je oprávnený každý člen tímu. Je ale potrebné aby sa označil ako “autor poslednej zmeny” a tiež aktualizoval “dátum poslednej zmeny” aby sme mali prehľad aký aktuálny je dokument. Ak je zmena zásadná, je nutné aby na ňu upozornil všetkých členov tímu, ktorí musia so zmenou súhlasiť. V prípade, že tím so zmenou nesúhlasí nie je možné zmenu zapracovať do metodiky.} \\\\
        \textnormal{Môže sa jednať o dva druhy zmien:}

        \begin{itemize}
            \item Malá zmena
            \begin{itemize}
                \item pod malou zmenou rozumieme zmenu, ktorá nemá zásadný vplyv na priebeh projektu a nijak významne nemení už doteraz vykonanú prácu.
                \item Postup pri zapracovaní malej zmeny:
                \begin{enumerate}
                    \item Definovanie zmeny v metodike
                    \item Predložiť návrh autorovi metodiky
                    \item Ak autor metodiky odsúhlasí zmenu, je možné vykonať zmenu
                    \item Ak autor metodiky neodsúhlasí zmenu, je možné prediskutovanie zmeny s tímom pomocou komunikačného kanálu
                    \item Po vykonaní zmeny v metodike je dobré informovanie tímu cez niektorý z komunikačných kanálov
                \end{enumerate}
            \end{itemize}

            \item Veľká zmena
            \begin{itemize}
                \item pod veľkou zmenou metodiky sa rozumie zmena, ktorá zásadne ovplyvňuje priebeh projektu alebo významne mení už doteraz vykonanú prácu
                \item Postup pri zapracovaní veľkej zmeny:
                \begin{enumerate}
                    \item Definovanie zmeny v metodike
                    \item Predložiť návrh zmeny tímu pomocou komunikačného kanálu
                    \item V prípade, že nie je potrebná hlbšia diskusia vytvoriť anketu
                    \item Na základe rozhodnutia tímu v ankete je možné vykonať/nevykonať zmenu metodiky
                \end{enumerate}
            \end{itemize}
        \end{itemize}

\end{document}
