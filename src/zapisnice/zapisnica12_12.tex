\documentclass{article}

\usepackage[margin=2cm]{geometry}
\usepackage[utf8]{inputenc}
\usepackage{hyperref}

% Title
\newcommand{\documenttitle}
{Zápisnica č. 12}

% Subtitle
\newcommand{\documentsubtitle}
{Stretnutie v tíme}

\begin{document}

    \newcommand{\fiitmark}{%
\large
\textnormal{%
Slovenská technická univerzita v Bratislave \\
\vspace*{.25cm}
Fakulta informatiky a informačných technológií} \\
}

\begin{titlepage}
        \begin{center}
            \vspace*{.5cm}
            \fiitmark

            \vfill
            \huge
            \textbf{\documenttitle} \\

            \vspace*{.25cm}
            \large
            \textbf{\documentsubtitle} \\

            \vspace*{.25cm}
            \textnormal{P3K Team (Tím 10)} \\

        \end{center}

        \vfill

        \begin{flushleft}
            \textbf{Vedúci Projektu}: Ing. Juraj Petrík \\
            \textbf{Akad. rok:} 2020/2021
        \end{flushleft}
\end{titlepage}


    \begin{table}[h]
        \begin{tabular}{lllll}
            \multicolumn{3}{l}{\textbf{Dátum stretnutia:}} & & 22.2.2021 \\
            \multicolumn{3}{l}{\textbf{Čas stretnutia:}} & & 16:00 - 18:00 hod. \\
            \multicolumn{3}{l}{\textbf{Prítomní:}} \\
            & & & & Bc. František Gič  \\
            & & & & Bc. Oliver Kanát \\
            & & & & Bc. Karin Maliniaková \\
            & & & & Bc. Denisa Mensatorisová \\
            & & & & Bc. Anton Rusňák \\
            & & & & Bc. Vladimír Svitok \\
            & & & & Bc. Kristián Toldy \\
            \multicolumn{3}{l}{\textbf{Neprítomní:}} & & -\\
            \multicolumn{3}{l}{\textbf{Zapisovateľ:}} & & Bc. Karin Maliniaková \\
        \end{tabular}
        \label{tab:grades}
    \end{table}

    \section*{Program:}

    \begin{enumerate}
        \item Standup
    \end{enumerate}

    \section*{Standup}

    \begin{itemize}
        \item \textbf {@fero}
        \begin{itemize}
            \item Práca na tasku, ktorý by mal naimplementovať pipeline v našom PW a FE repe, ktoré by vybuildili a uploadli subory na nas produkcny web, aby sme to nemuseli robit manualne.
            \item Začal lokálne robiť na tej pipeline, číta rozdiely medzi gitlab a github pipelines, kedže skúsenosť má zatiaľ len v gitlabe.
        \end{itemize}
        \item \textbf {@tono}
        \begin{itemize}
            \item Robí na tasku implementácie spinner-u a fetch-u na result page, zatiaľ veľmi nepokročil
        \end{itemize}
        \item \textbf {@dena}
        \begin{itemize}
            \item Práca na responzitive
            \item page Home a Upload sú hotové, ostáva ešte Result a Document page
        \end{itemize}
        \item \textbf {@karin}
        \begin{itemize}
            \item práca na Parallel language checking
            \item nejasnosti v zadaní úlohy
        \end{itemize}
        \item \textbf {@vlado}
        \begin{itemize}
            \item práca na greedy string tiling
            \item má hotový greedy string tiling avšak potrebuje dohodnúť pár vecí - na to porovnavanie, od akej dĺžky zhody znakov chceme aby sa to rátalo ako podobnosť, ak je string 'abcdef' a string 'deabcf', keď dám najkratší možný 2 tak by vezme len 'abc' a 'de'. treba nejaku rozumnu hodnotu na to
            \item ďalej je potrebné tie intervali - aké su tieto podobnosti kam sa vlastne maju ukladať, ako sa má počítať tá podobnosť (napr. zoberiem dĺžku toho podobneho textu a len vypočítam koľko to je %  z celeho textu?
        \end{itemize}
        \item \textbf {@oliver}
        \begin{itemize}
            \item práca na dokerizácii, zatiaľ žiadne výsledky
        \end{itemize}
        \item \textbf {@kiko}
        \begin{itemize}
            \item práca na limitovaní Upload súborov
            \item zisťovanie čo treba spraviť - kvôli bezpečnosti treba requesty handlovať aj na backende a kvôli použiteľnosti urobiť aj na frontende chybové hlášky
            \item zatiaľ teda nemá žiadny problém, ale bude potrebovať neskôr pomôcť s testami - ako posielať testovacie requesty
        \end{itemize}
    \end{itemize}    

\end{document}
