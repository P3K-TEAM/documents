\documentclass{article}

\usepackage[margin=2cm]{geometry}
\usepackage[utf8]{inputenc}
\usepackage{hyperref}

% Title
\newcommand{\documenttitle}
{Zápisnica č. 8}

% Subtitle
\newcommand{\documentsubtitle}
{Stretnutie s vedúcim}

\begin{document}

    \newcommand{\fiitmark}{%
\large
\textnormal{%
Slovenská technická univerzita v Bratislave \\
\vspace*{.25cm}
Fakulta informatiky a informačných technológií} \\
}

\begin{titlepage}
        \begin{center}
            \vspace*{.5cm}
            \fiitmark

            \vfill
            \huge
            \textbf{\documenttitle} \\

            \vspace*{.25cm}
            \large
            \textbf{\documentsubtitle} \\

            \vspace*{.25cm}
            \textnormal{P3K Team (Tím 10)} \\

        \end{center}

        \vfill

        \begin{flushleft}
            \textbf{Vedúci Projektu}: Ing. Juraj Petrík \\
            \textbf{Akad. rok:} 2020/2021
        \end{flushleft}
\end{titlepage}


    \begin{table}[h]
        \begin{tabular}{lllll}
            \multicolumn{3}{l}{\textbf{Dátum stretnutia:}} & & 20.11.2020 \\
            \multicolumn{3}{l}{\textbf{Čas stretnutia:}} & & 16:00 - 19:00 hod. \\
            \multicolumn{3}{l}{\textbf{Prítomní:}} \\
            & & vedúci: & & Ing. Juraj Petrík \\
            & & členovia tímu: & & Bc. František Gič  \\
            & & & & Bc. Oliver Kanát \\
            & & & & Bc. Karin Maliniaková \\
            & & & & Bc. Denisa Mensatorisová \\
            & & & & Bc. Anton Rusňák \\
            & & & & Bc. Vladimír Svitok \\
            & & & & Bc. Kristián Toldy \\
            \multicolumn{3}{l}{\textbf{Neprítomní:}} & & -\\
            \multicolumn{3}{l}{\textbf{Zapisovateľ:}} & & Bc. Oliver Kanát \\
        \end{tabular}
        \label{tab:grades}
    \end{table}

    \section*{Program:}
    
    \begin{enumerate}
        \item Sprint review 
        \item Retrospektíva
    \end{enumerate}

    cielom je uploadnut subory alebo text a na serveri sa vygeneruje IDE do submitionu a ten sa vrati ako response. potom sa to redirektne na ruselt pagu podla IDE, na result pagi sa to bude tocit a ked bude vysledok tak sa to zobrazi.
    

\end{document}
