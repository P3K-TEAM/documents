\documentclass{article}

\usepackage[margin=2cm]{geometry}
\usepackage[utf8]{inputenc}
\usepackage{hyperref}

% Title
\newcommand{\documenttitle}
{Zápisnica č. 8}

% Subtitle
\newcommand{\documentsubtitle}
{Stretnutie s vedúcim}

\begin{document}

    \newcommand{\fiitmark}{%
\large
\textnormal{%
Slovenská technická univerzita v Bratislave \\
\vspace*{.25cm}
Fakulta informatiky a informačných technológií} \\
}

\begin{titlepage}
        \begin{center}
            \vspace*{.5cm}
            \fiitmark

            \vfill
            \huge
            \textbf{\documenttitle} \\

            \vspace*{.25cm}
            \large
            \textbf{\documentsubtitle} \\

            \vspace*{.25cm}
            \textnormal{P3K Team (Tím 10)} \\

        \end{center}

        \vfill

        \begin{flushleft}
            \textbf{Vedúci Projektu}: Ing. Juraj Petrík \\
            \textbf{Akad. rok:} 2020/2021
        \end{flushleft}
\end{titlepage}


    \begin{table}[h]
        \begin{tabular}{lllll}
            \multicolumn{3}{l}{\textbf{Dátum stretnutia:}} & & 20.11.2020 \\
            \multicolumn{3}{l}{\textbf{Čas stretnutia:}} & & 16:00 - 19:00 hod. \\
            \multicolumn{3}{l}{\textbf{Prítomní:}} \\
            & & vedúci: & & Ing. Juraj Petrík \\
            & & členovia tímu: & & Bc. František Gič  \\
            & & & & Bc. Oliver Kanát \\
            & & & & Bc. Karin Maliniaková \\
            & & & & Bc. Denisa Mensatorisová \\
            & & & & Bc. Anton Rusňák \\
            & & & & Bc. Vladimír Svitok \\
            & & & & Bc. Kristián Toldy \\
            \multicolumn{3}{l}{\textbf{Neprítomní:}} & & -\\
            \multicolumn{3}{l}{\textbf{Zapisovateľ:}} & & Bc. Oliver Kanát \\
        \end{tabular}
        \label{tab:grades}
    \end{table}

    \section*{Program:}
    
    \begin{enumerate}
        \item Sprint review 
        \item Retrospektíva
    \end{enumerate}

    \section*{Sprint Review}

        \textnormal {Pri viacerćy úlohách sa ukázalo, že nie všetky požiadavky boli správne pochopené a preto riešsenie je síce funkčné ale nie je na 100\% podľa predstáv. }

    \section*{Plánovanie šprintu}

    \textnormal {Cieľom šprintu je uploadnuť súbory alebo text, následne na serveri sa vygeneruje IDE do submission a ten sa vrati ako response. Následne sa vytvorí redirekt na result page podľa IDE, kde sa to bude načítavať a keď bude známy výsledok tak sa to zobrazí.}

    \begin{itemize}
    \item \textbf {PW - Prezentačný web} 
        \begin{itemize}
            \item PW32 - Export úloh z Jiry \textbf {@karin}
            \item PW33 - Napísať moduly systému do dokumentácie inžinierskeho diela \textbf {@karin}
        \end{itemize}
    \item \textbf {FRON - Frontend} 
        \begin{itemize}
            \item FRON22 - Unit testy \textbf {@fero}
        \end{itemize}  
    \item \textbf {BE - Backend} 
        \begin{itemize}
            \item BE41 - Implement text upload endpoint (BE40 - Compare documents) \textbf {@vlado}
            \item BE42 - Create background task for text and file comparison (BE40 - Compare documents) \textbf {@kiko}
            \item BE46 - Implement universal document model (BE40 - Compare documents) \textbf {@oliver}
            \item BE47 - Update existing files upload to use new document model (BE40 - Compare documents) \textbf {@vlado}
            \item BE44 - Implement text extraction method for different types of text files (BE43 - Text extraction from files) \textbf {@dena}
            \item BE45 - Implement text extraction method from image files (OCR) (BE43 - Text extraction from files) \textbf {@karin}
            \item BE49 - Find out what the heck Juro's done (BE48 - Juro's crawlers) \textbf {@tono}
            \item BE50 - Push data from crawler to Elasticsearch (BE48 - Juro's crawlers) \textbf {@tono}
            \item BE52 - Import wiki to Elasticsearch (BE51 - Playing with corpus) \textbf {@fero}
            \item BE53 - Import referaty.sk to Elasticsearch (BE51 - Playing with corpus) \textbf {@fero}
            \item BE54 - Integrate Elasticsearch with django (BE51 - Playing with corpus) \textbf {@oliver}
            \item BE55 - Search for similar documents in Elastic (BE51 - Playing with corpus) \textbf {@oliver}
        \end{itemize}        
    \end{itemize}

    \section*{Poznámky zo stretnutia:}

    \begin{itemize}
        \item Vytvoriť meeting, kde sa navhrnú endpointy a štruktúra databázy.  
    \end{itemize}    
\end{document}
