\documentclass{article}

\usepackage[margin=2cm]{geometry}
\usepackage[utf8]{inputenc}
\usepackage{hyperref}

% Title
\newcommand{\documenttitle}
{Zápisnica č. 2}

% Subtitle
\newcommand{\documentsubtitle}
{Stretnutie s vedúcim}

\begin{document}

    \newcommand{\fiitmark}{%
\large
\textnormal{%
Slovenská technická univerzita v Bratislave \\
\vspace*{.25cm}
Fakulta informatiky a informačných technológií} \\
}

\begin{titlepage}
        \begin{center}
            \vspace*{.5cm}
            \fiitmark

            \vfill
            \huge
            \textbf{\documenttitle} \\

            \vspace*{.25cm}
            \large
            \textbf{\documentsubtitle} \\

            \vspace*{.25cm}
            \textnormal{P3K Team (Tím 10)} \\

        \end{center}

        \vfill

        \begin{table}[h]
            \begin{tabular}{ll}
                \textbf{Vedúci projektu:} & Ing. Juraj Petrík \\
                \textbf{Ak. rok:}  & 2020/2021 \\
            \end{tabular}
            \label{tab:grades}
        \end{table}
\end{titlepage}


    \begin{table}[h]
        \begin{tabular}{lllll}
            \multicolumn{3}{l}{\textbf{Dátum stretnutia:}} & & 9.10.2020 \\
            \multicolumn{3}{l}{\textbf{Čas stretnutia:}} & & 16:00 - 19:00 hod. \\
            \multicolumn{3}{l}{\textbf{Prítomní:}} \\
            & & vedúci: & & Ing. Juraj Petrík \\
            & & členovia tímu: & & Bc. František Gič  \\
            & & & & Bc. Oliver Kanát \\
            & & & & Bc. Karin Maliniaková \\
            & & & & Bc. Denisa Mensatorisová \\
            & & & & Bc. Anton Rusňák \\
            & & & & Bc. Vladimír Svitok \\
            & & & & Bc. Kristián Toldy \\
            \multicolumn{3}{l}{\textbf{Neprítomní:}} & & -\\
            \multicolumn{3}{l}{\textbf{Zapisovateľ:}} & & Bc. Karin Maliniaková \\
        \end{tabular}
        \label{tab:grades}
    \end{table}

    \section*{Program:}

    \begin{enumerate}
        \item Sprint review
        \item Retrospektíva šprintu
        \item Odhadovanie položiek Produktového backlogu
        \item Plánovanie šprintu
    \end{enumerate}

    \section*{Sprint Review}

    \begin{itemize}
        \item \underline {PW6 - Vytvoriť šablónu zápisnice} - šablóna je vytvorená vo Worde a dohodli sme sa, že šablónu na web nebudeme zverejňovať
        \item \underline {PW8 - Dizajn: Zvoliť tému alebo vlastné riešenie} - spísať celú analýzu (medzi čím sme sa dohadovali a k čomu sme došli, nech je z toho nejaký výstup), následne zverejniť na Conflence
        \item \underline {PW5 - Zohnať informácie o členoch tímu} - Informácie/fotky pripravené na použitie pri vytváraní prezentačného webu - uložené na Google disku
        \item \underline {PW4 - Vytvoriť GitHub repozitár} - vytvorený
        \item \underline {BE1 - Analýza zdrojov údajov} - úloha sa v priebehu šprintu rozdelila na 3 samostatné tasky
        \begin{itemize}
            \item BE15 - Pozbierať outcomes z prvého meetingu s vedúcim
            \item BE16 - Dohodnúť a napísať zdroje údajov, ktoré budeme čerpať
            \item BE17 - Spraviť research ohľadom extrahovania textu z jednotlivých zdrojov (oskole, referaty - či majú API)
            \item nedostatočne splnené všetky tri tasky - treba ku všetkému aj nejaký dokument, kde to bude všetko jasne popísané a preskúmať viac do hĺbky
        \end{itemize}
    \end{itemize}

    \section*{Retrospektíva šprintu}

    \begin{itemize}
    \item \textbf {Čo sa nám páčilo/podarilo:} 
        \begin{itemize}
            \item väčšina taskov bola doručená aj napriek neskorému naplánovaniu šprintu
        \end{itemize}
    \item \textbf {Čo sa nám nepáčilo/nepodarilo:}
        \begin{itemize}
            \item neskoré plánovanie šprintu
            \item nedostatočný opis taskov
        \end{itemize}
    \item \textbf {Čo by sme mali zaviesť:} 
        \begin{itemize}
            \item plánovanie šprintu skôr - na stretnutí s vedúcim, kedy ukončujeme šprint
            \item podrobnejší opis taskov
        \end{itemize}
    \end{itemize}


    \section*{Plánovanie šprintu}

    \begin{itemize}
    \item \textbf {PW - Prezentačný web} 
        \begin{itemize}
            \item PW9 - Vytvoriť základ podstránky, založiť projekt \textbf {@fero}
            \item PW10 - Vytvoriť časť o členoch tímu \textbf {@tono}
            \item PW11 - Vytvoriť časť o dokumentoch \textbf {@karin}
            \item PW12 - Vytvoriť kontaktnú časť \textbf {@dena}
            \item PW13 - Vytvoriť časť "O projekte" \textbf {@kiko}
            \item PW14 - Vytvoriť navigáciu \textbf {@fero}
            \item PW15 - Aktualizácia dát \textbf {@karin}
            \item PW16 - Vytvoriť dizajn tímového webu v Adobe XD \textbf {@dena}   
        \end{itemize}
    \item \textbf {BE - Backend} 
        \begin{itemize}
            \item BE18 - Prirodzený text (BE1 - Analýza zdrojov údajov) \textbf {@kiko}
            \item BE19 - Zdrojový kód (BE1 - Analýza zdrojov údajov) \textbf {@dena}
            \item BE20 - Pre SK (BE2 - Analýza predspracovania textu) \textbf {@fero}
            \item BE21 - Pre ENG (BE2 - Analýza prespracovania textu) \textbf {@vlado}
            \item BE23 - vektorové metódy (BE5 - Analýza porovnávania podobnosti textu) \textbf {@oliver}
            \item BE24 - štatistické metódy/citačné metódy (Be5 - Analýza porovnávania podobnosti textu \textbf {@tono}
            \item BE25 - Antiplag nástroje \textbf {@karin}
        \end{itemize}        
    \end{itemize}

    \section*{Poznámky zo stretnutia:}

    \begin{itemize}
        \item Vytvoriť Confluence - na písanie dokumentácií, analýz, metodík... 
        \item Dohodli sme sa, že pri plánovaní šprintu budeme brať do úvahy, že osoba má na 2 týždne - 2MD. Čiže na dvojtýždňový šprint môžeme ako tím zobrať úlohy ohodnotené 14MD.
        \item Prebehli sprint review a retrospektíva šprintu.
        \item Šprint je zdokumentovaný.
        \item Šprint bol akceptovaný vlastníkom produktu.
    \end{itemize}

\end{document}
