\documentclass{article}

\usepackage[margin=2cm]{geometry}
\usepackage[utf8]{inputenc}
\usepackage{hyperref}

% Title
\newcommand{\documenttitle}
{Zápisnica č. 3}

% Subtitle
\newcommand{\documentsubtitle}
{Stretnutie s vedúcim}

\begin{document}

    \newcommand{\fiitmark}{%
\large
\textnormal{%
Slovenská technická univerzita v Bratislave \\
\vspace*{.25cm}
Fakulta informatiky a informačných technológií} \\
}

\begin{titlepage}
        \begin{center}
            \vspace*{.5cm}
            \fiitmark

            \vfill
            \huge
            \textbf{\documenttitle} \\

            \vspace*{.25cm}
            \large
            \textbf{\documentsubtitle} \\

            \vspace*{.25cm}
            \textnormal{P3K Team (Tím 10)} \\

        \end{center}

        \vfill

        \begin{table}[h]
            \begin{tabular}{ll}
                \textbf{Vedúci projektu:} & Ing. Juraj Petrík \\
                \textbf{Ak. rok:}  & 2020/2021 \\
            \end{tabular}
            \label{tab:grades}
        \end{table}
\end{titlepage}


    \begin{table}[h]
        \begin{tabular}{lllll}
            \multicolumn{3}{l}{\textbf{Dátum stretnutia:}} & & 16.10.2020 \\
            \multicolumn{3}{l}{\textbf{Čas stretnutia:}} & & 16:00 - 19:00 hod. \\
            \multicolumn{3}{l}{\textbf{Prítomní:}} \\
            & & vedúci: & & Ing. Juraj Petrík \\
            & & členovia tímu: & & Bc. František Gič  \\
            & & & & Bc. Oliver Kanát \\
            & & & & Bc. Karin Maliniaková \\
            & & & & Bc. Denisa Mensatorisová \\
            & & & & Bc. Anton Rusňák \\
            & & & & Bc. Vladimír Svitok \\
            & & & & Bc. Kristián Toldy \\
            \multicolumn{3}{l}{\textbf{Neprítomní:}} & & -\\
            \multicolumn{3}{l}{\textbf{Zapisovateľ:}} & & Bc. Karin Maliniaková \\
        \end{tabular}
        \label{tab:grades}
    \end{table}

    \section*{Program:}

    \begin{enumerate}
        \item Standup
        \item Doplnenie Produktového backlogu
        \item Odhadovanie položiek Produktového backlogu
        \item Diskusia
    \end{enumerate}

    \section*{Poznámky zo stretnutia:}

    \begin{itemize}
        \item Prešli sme si stav úloh
        \begin{itemize}
            \item viac sa pracovalo na tímovom prezentačnom webe
            \item na druhú časť šprintu zostávajú úlohy z backendu
            \item vyzerá to, že šprint by sme mohli stihnúť
        \end{itemize}
        \item Do Produktového backlogu sme doplnili ďalšie tasky a zadefinovali ich opis
        \item Odhadovali sme tasky s najvyššou prioritou v Produktovom backlogu, aby sme ušetrili čas na ďalšom stretnutí, kde budeme plánovať nový šprint - využili sme metódu "Plánovacieho pokru"
    \end{itemize}


    

\end{document}
