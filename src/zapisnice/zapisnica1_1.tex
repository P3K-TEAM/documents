\documentclass{article}

\usepackage[margin=2cm]{geometry}
\usepackage[utf8]{inputenc}
\usepackage{hyperref}

% Title
\newcommand{\documenttitle}
{Zápisnica č. 1}

% Subtitle
\newcommand{\documentsubtitle}
{Stretnutie v tíme}

\begin{document}

    \newcommand{\fiitmark}{%
\large
\textnormal{%
Slovenská technická univerzita v Bratislave \\
\vspace*{.25cm}
Fakulta informatiky a informačných technológií} \\
}

\begin{titlepage}
        \begin{center}
            \vspace*{.5cm}
            \fiitmark

            \vfill
            \huge
            \textbf{\documenttitle} \\

            \vspace*{.25cm}
            \large
            \textbf{\documentsubtitle} \\

            \vspace*{.25cm}
            \textnormal{P3K Team (Tím 10)} \\

        \end{center}

        \vfill

        \begin{flushleft}
            \textbf{Vedúci Projektu}: Ing. Juraj Petrík \\
            \textbf{Akad. rok:} 2020/2021
        \end{flushleft}
\end{titlepage}


    \begin{table}[h]
        \begin{tabular}{lllll}
            \multicolumn{3}{l}{\textbf{Dátum stretnutia:}} & & 7.10.2020 \\
            \multicolumn{3}{l}{\textbf{Čas stretnutia:}} & & 12:00 - 14:00 hod. \\
            \multicolumn{3}{l}{\textbf{Prítomní:}} \\
            & & & & Bc. František Gič  \\
            & & & & Bc. Karin Maliniaková \\
            & & & & Bc. Denisa Mensatorisová \\
            & & & & Bc. Anton Rusňák \\
            & & & & Bc. Vladimír Svitok \\
            & & & & Bc. Kristián Toldy \\
            \multicolumn{3}{l}{\textbf{Neprítomní:}} & &  Bc. Oliver Kanát - návšteva lekára \\\\
            \multicolumn{3}{l}{\textbf{Zapisovateľ:}} & & Bc. František Gič \\
        \end{tabular}
        \label{tab:grades}
    \end{table}

    \section*{Agenda:}

    \begin{enumerate}
        \item Piatok končí šprint - treba naplánovať šprint
        \item Do konca októbra treba mať spravený web tímu
        \begin{itemize}
            \item založiť git
            \item šablóna zápisnice
            \item nadizajnovať (alias rozhodnúť sa)
            \begin{itemize}
                \item CSS Framework vs. Reaktívny framework
            \end{itemize}
        \end{itemize}
        \item Treba začať aj s prácou na produkte (implementácia alebo analýza) 
    \end{enumerate}

    \section*{Plánovanie šprintu}

    \begin{itemize}
    \item \textbf {PW - Prezentačný web} 
        \begin{itemize}
            \item PW4 - Vytvoriť GitHub repozitár \textbf {@fero}
            \item PW5 - Zohnať informácie o členoch tímu \textbf {@dena}
            \item PW6 - Vytvoriť šablónu zápisnice \textbf {@karin}
            \item PW8 - Dizajn: Zvoliť tému alebo vlastné riešenie \textbf {@tono}   
        \end{itemize}
    \item \textbf {BE - Backend} 
        \begin{itemize}
            \item BE1 - BE1 - Analýza zdrojov údajov \textbf {@kiko (@vlado, @oliver)}
        \end{itemize}        
    \end{itemize}

\end{document}
